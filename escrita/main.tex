\documentclass[
12pt,        % tamanho da fonte
openright,   % capitulos comecam em paginas impares, insere paginas em branco se necessario
twoside,     % para impressao frente e verso, comente esta linha se for imprimir só frente.
a4paper,     % tamanho do papel
% -- opções da classe abntex2 -- retire o comentario para obter o comportamento
% chapter=TITLE,         % títulos de capítulos convertidos em letras maiúsculas
% section=TITLE,         % títulos de seções convertidos em letras maiúsculas
% subsection=TITLE,      % títulos de subseções convertidos em letras maiúsculas
% subsubsection=TITLE,% títulos de subsubseções convertidos em letras maiúsculas
% -- opções do pacote polyglossia --
% french,      % idioma adicional para hifenizacao
% spanish,     % idioma adicional para hifenizacao
brazil,       % idioma adicional para hifenizacao
english       % ultimo idioma eh o principal do documento
%
% ppgca.cls options
%
%,englishwr      % For documents written in english, remove only the coment '%'
]{ppgca}

%%%%%%%%%%%%%%%%%%%%%%%%%%%%%%%%%%%%%%%%%%%%%%%%%%%%%%%%%%%%%%%%%%%%%
% Para que o primeiro parágrafo também seja 'indentado':
% troque \ifnum1=0 por \ifnum1=1
%%%%%%%%%%%%%%%%%%%%%%%%%%%%%%%%%%%%%%%%%%%%%%%%%%%%%%%%%%%%%%%%%%%%%
\ifnum1=0
\ifxetexorluatex
\PolyglossiaSetup{brazil}{indentfirst=true}
\PolyglossiaSetup{english}{indentfirst=true}
\else
\usepackage{indentfirst}
\fi
\fi

%%%%%%%%%%%%%%%% VERSÃO DO DOCUMENTO: ORIGINAL OU CORRIGIDA
% Após as correções sugeridas pela banca serem efetuadas, retire os comentários
% da próxima linha.
%\versaodocumento{corrigida}

% Este arquivo foi baseado no modelo disponível em https://www.ctan.org/pkg/abntex2.

% Para gerar o indice, execute o comando makeindex:
% makeindex main

% O preambulo deve conter o tipo do trabalho, o objetivo,
% o nome da instituição e a área de concentração
\preambulo{Final Paper for the Bachelor in Computer Science program under Departamento de Computação e Matemática of the Universidade de São Paulo}

\usepackage{blindtext}

%---
% Informações de dados para CAPA e FOLHA DE ROSTO
% ---
\title{An IoT Smart Scale Proof of Concept for Smart Homes}
\author{Gabriel Carvalho Silva}
\local{Ribeirão Preto--SP}
\data{2025}
\orientador{Cléver Ricardo Guareis de Farias}
\coorientador{}
\tipotrabalho{Final Paper} % Dissertação ou Tese

% ---
% Espaçamentos entre linhas e parágrafos
% ---
% O tamanho do parágrafo é dado por:
\setlength{\parindent}{1.3cm}

% Controle do espaçamento entre um parágrafo e outro:
\setlength{\parskip}{0.2cm}  % tente também \onelineskip

%#% you can change the language used (brazil) by set and uncomment the
%#% following command:
% \setdefaultlanguage{english}

% #% Options for the \setdefaultlanguage{} can be found at
% #% http://mirrors.ctan.org/macros/latex/contrib/polyglossia/polyglossia.pdf#page=5

% ---
% ---
% compila o indice
% ---
\makeindex
% ---
\begin{document}
%%%%%%%%%%%%%%%%%%%%%%%%%%%%%%%%%%%%%%%%%%%%%%%%%%%%%%%%%%%%%%%%%%%%%
% Para limpar o cabeçalho, troque \ifnum1=0 por \ifnum1=1
%%%%%%%%%%%%%%%%%%%%%%%%%%%%%%%%%%%%%%%%%%%%%%%%%%%%%%%%%%%%%%%%%%%%%
\ifnum1=0
\newcommand{\sectionbreak}{\clearpage
\fancyhead[LE,RO]{}
\fancyhead[RE,LO]{}
\renewcommand{\headrulewidth}{0pt}
\renewcommand{\footrulewidth}{0pt}}
\fi

% ----------------------------------------------------------
% ELEMENTOS PRÉ-TEXTUAIS
% ----------------------------------------------------------
% \pretextual

% ---
% Capa
% ---
\imprimircapa
% ---

% ---
% Folha de rosto
% (o * indica que haverá a ficha bibliográfica)
% ---
% \imprimirfolhaderosto
% ---

%---
% Imprime a folha de rosto em inglês (Opcional)
\coversheet{An IoT Smart Scale Proof of Concept for Smart Homes}
%---


% ---

% ---
% Inserir errata
% ---
% \begin{errata}
% Elemento opcional da NBR14724:2011. Exemplo:

% \vspace{\onelineskip}

% FERRIGNO, C. R. A. \textbf{Tratamento de neoplasias ósseas apendiculares com
% reimplantação de enxerto ósseo autólogo autoclavado associado ao plasma
% rico em plaquetas}: estudo crítico na cirurgia de preservação de membro em
% cães. 2011. 128 f. Tese (Livre-Docência) - Faculdade de Medicina Veterinária e
% Zootecnia, Universidade de São Paulo, São Paulo, 2011.

% \begin{table}[htb]
% \center
% \footnotesize
% \begin{tabular}{|p{1.4cm}|p{1cm}|p{3cm}|p{3cm}|}
%   \hline
%    \textbf{Folha} & \textbf{Linha}  & \textbf{Onde se lê}  & \textbf{Leia-se}  \\
%     \hline
%     1 & 10 & auto-conclavo & autoconclavo\\
%    \hline
% \end{tabular}
% \end{table}

% \end{errata}
% ---

% ---
% Inserir folha de aprovação
% ---

% Isto é um exemplo de Folha de aprovação, elemento obrigatório da NBR
% 14724/2011 (seção 4.2.1.3). Você pode utilizar este modelo até a aprovação
% do trabalho. Após isso, substitua todo o conteúdo deste arquivo por uma
% imagem da página assinada pela banca com o comando abaixo:
%
% \begin{folhadeaprovacao}
% \includepdf{folhadeaprovacao_final.pdf}
% \end{folhadeaprovacao}
%
% \begin{folhadeaprovacao}

%   \begin{center}
%     {\theauthor}

%     \vspace*{\fill}\vspace*{\fill}
%     \thetitle
%     \vspace*{\fill}

%     \hspace{.45\textwidth}
%     \begin{minipage}{.5\textwidth}
%         \imprimirpreambulo
%     \end{minipage}%
%     \vspace*{\fill}
%    \end{center}

%    Trabalho aprovado. \imprimirlocal, 21 de novembro de 2018:

%    \assinatura{\textbf{\thesupervisorlabel:} \\ Orientador}
%    \assinatura{\textbf{Professor} \\ Convidado 1}
%    \assinatura{\textbf{Professor} \\ Convidado 2}
%    \ifthenelse{\equal{\imprimirtipotrabalho}{Tese}}{
%      \assinatura{\textbf{Professor} \\ Convidado 3}
%      \assinatura{\textbf{Professor} \\ Convidado 4}
%    }{}

%    \begin{center}
%     \vspace*{0.5cm}
%     {\large\imprimirlocal}
%     \par
%     {\large\imprimirdata}
%     \vspace*{1cm}
%   \end{center}

% \end{folhadeaprovacao}
% ---

% ---
% Dedicatória
% ---
\begin{dedicatoria}
   \vspace*{\fill}
   \centering
   \noindent
   \textit{A minha avó, Maria Daria Rocha.} \vspace*{\fill}
\end{dedicatoria}
% ---

% ---
% Agradecimentos
% ---
\begin{agradecimentos}
Agradeço $\ldots$
\end{agradecimentos}
% ---

% ---
% Epígrafe
% ---
\begin{epigrafe}
    \vspace*{\fill}
        \begin{flushright}
                \textit{``RAAAWR //
                          (Totoro)}
        \end{flushright}
\end{epigrafe}
% ---

% ---
% RESUMOS
% ---

% resumo em português
% remover se o documento for em inglês
% \setlength{\absparsep}{18pt} % ajusta o espaçamento dos parágrafos do resumo
% \begin{resumo}
%   Este documento é um modelo \LaTeX para servir como base para edição
%   de uma dissertação a ser apresentada ao programa de pós-graduação em
%   Computação Aplicada do Departartamento de Computação e Matemática da
%   FFCLRP/USP.

% \noindent \textbf{Palavras-chave}: latex. abntex. editoração de texto.
% \end{resumo}

% resumo em inglês
\begin{resumo}[Abstract]
% \begin{otherlanguage*}{english}
   This is the english abstract.

   \vspace{\onelineskip}

   \noindent \textbf{Keywords}: iot. smart home. event driven. face recognition. smart scale.
% \end{otherlanguage*}
\end{resumo}

% OBS: A numeração de páginas deve sempre começar em páginas ímpares,
% por isto o uso de \cleardoublepage.

% ---
% inserir lista de figuras
% ---
\pdfbookmark[0]{\listfigurename}{lof}
\listoffigures*
\cleardoublepage
% ---

% ---
% inserir lista de quadros (opcional)
% ---
% \pdfbookmark[0]{\listofquadrosname}{loq}
% \listofquadros*
% \cleardoublepage
% ---

% ---
% inserir lista de tabelas
% ---
\pdfbookmark[0]{\listtablename}{lot}
\listoftables*
\cleardoublepage
% ---

% ---
% inserir lista de abreviaturas e siglas
% ---
\begin{siglas}
  \item[TODO] TODO
\end{siglas}
% ---

% ---
% inserir lista de símbolos
% ---
\begin{simbolos}
  \item[$ \Gamma $] TODO
\end{simbolos}
% ---

% ---
% inserir o sumario
% ---
\pdfbookmark[0]{\contentsname}{toc}
\tableofcontents*
\cleardoublepage
% ---
% ----------------------------------------------------------
% ELEMENTOS TEXTUAIS
% ----------------------------------------------------------
\textual

\chapter{Introduction} % capítulo não numerado, SEM asterisco
\section{Background}
\section{Objective}
\section{Methodology}
\section{Structure of this paper}


\chapter{Literature and state of the art review}
\section{Internet of Things}
\subsection{Communication of Things}
\subsection{Privay in IoT}
\section{Smart Homes}
\section{Smart Scales}
\section{Related Work}


\chapter{Proof of Concept}
\section{Project specification}
\subsection{Functional Requirements}
\subsection{Nonfunctional Requirements}
\section{Project Design}
\subsubsection{Infrastructure}
ESP32 and equipment
raspberry pi placeholded by laptop (docker)
\subsubsection{System architecture}
DDD and event driven
\subsubsection{System communication}
Data serialization
\subsection{Design of the edge layer}
\subsection{Design of the fog layer}
\subsection{Design of the cloud/hub layer}
\section{Smart Scale Face Recognition}
\subsection{Method comparison}
Euclidian distance X cosine \& neural net vs basic moment calculation
\subsection{Handling Privacy}
\section{Project Evaluation}
\subsection{Processing Performance}
\subsection{Usability Performance}

\phantompart

% ---
% Conclusão
% ---
\chapter{Conclusion}
\section{Contributions}
\section{Discussion}
\section{Future work}
\subsection{Beyond the PoC}
\subsection{Applications to the Internet of Medical Things}
\subsection{Applications to Husbandry}
% ---



% ----------------------------------------------------------
% ELEMENTOS PÓS-TEXTUAIS
% ----------------------------------------------------------
% Retire o comentário somente se o padrão exigir que daqui para a
% frente não haja número de páginas.
%\postextual
% ----------------------------------------------------------

% ------
\bibliography{refs}
% ------

% ----------------------------------------------------------
% Glossário
% ----------------------------------------------------------
%
% Consulte o manual da classe abntex2 para orientações sobre o glossário.
%
%\glossary

% ----------------------------------------------------------
% Apêndices
% ----------------------------------------------------------

% ---
% Inicia os apêndices
% ---
\begin{apendicesenv}

% Imprime uma página indicando o início dos apêndices
\partapendices

% ----------------------------------------------------------
\chapter{Quisque libero justo}
% ----------------------------------------------------------



\vfill

\pagebreak



% ----------------------------------------------------------
\chapter{Nullam elementum}
% ----------------------------------------------------------


\end{apendicesenv}
% ---

% --% ---
\begin{anexosenv}

% Imprime uma página indicando o início dos anexos
\partanexos

% ---
\chapter{Morbi ultrices rutrum lorem.}
% ---


% ---
\chapter{Fusce facilisis lacinia dui}
% ---



\end{anexosenv}

%---------------------------------------------------------------------
% INDICE REMISSIVO
%---------------------------------------------------------------------
\phantompart
\printindex
%---------------------------------------------------------------------
\end{document}
